\documentclass[letterpaper,10pt,draftclsnofoot,onecolumn,titlepage]{IEEEtran}

\usepackage{graphicx}
\usepackage{amssymb}
\usepackage{amsmath}
\usepackage{amsthm}
\usepackage{alltt}
\usepackage{float}
\usepackage{color}
\usepackage{url}
\usepackage{enumitem}
\usepackage{pstricks, pst-node}
\usepackage{geometry}
\usepackage{longtable}

\geometry{margin = .75in}

\def\name{Kevin Stine}

\begin{document}
	\title{\huge Writing Topic 3: Interrupts\\CS 444 Fall 2016}
	\author{\large \name}
	\maketitle
	\newpage
	\section*{Interrupts}
		Modern operating systems are complex beasts comprised of various functionality such as I/O, process scheduling, file systems, interrupts and synchronization.
	  In operating systems, an interrupt is a signal to the processor that is emitted by hardware or software indicating an event that needs immediate attention.
		Interrupts are synchronous with respect to the current process meaning that interrupts can happen at any time.
		Most computers have many devices such as a keyboard, mouse, hard drive, printer and scanners plugged into them.
		These various devices all need to have access to the CPU, however we are unable to predict exactly when it will need that access.
		Because of this, modern operating systems contain interrupts, which are a way for the CPU to find out which devices need attention and when.
		Interrupts work by giving each device an interrupt line that it can use to signal the processor.
		When the interrupt is signaled, the processor executes a routine called an interrupt handler to deal with the interrupt.
		The benefit of interrupts is there is no overhead when there are no requests pending.
		Despite most operating systems using interrupts, their implementation for interrupts and interrupt handling differs between each operating system.

		FreeBSD follows the pattern of other multi-threaded Unix kernels and deals with interrupt handles by giving them their own thread context.
		Giving an interrupt handler it's own thread context allows for the interrupt handler to block on locks and are run at real-time kernel priority.
		In FreeBSD, the interrupt threads are referred to as heavyweight interrupt threads because switching to an interrupt thread involves a full context switch.
		Currently, the kernel in FreeBSD is preemptive, however they only preempt a kernel thread when they release a sleep mutex or when an interrupt comes in.
		Having a fully preemptive kernel is in the pipeline for the future FreeBSD kernel.

		Windows utilizes the Windows Driver Framework (WDF) which allows for drivers to handle a devices hardware interrupts.
		In order for a WDF driver to handle a device's hardware interrupts, it must first create a framework object for each interrupt that each device can support.
		In the most recent versions of Windows (8 and newer), Kernel-Mode Driver Frameworks (KMDF) and User-Mode Driver Frameworks (UMDF) drivers can create interrupt objects requiring passive-level handling.
		Passive-level handling is used in specific cases such as System on a Chip (SoC) platforms, but most drivers should utilize DIRQL interrupt objects.


	\clearpage

	\begin{thebibliography}{3}
		\bibitem{FreeBSD}
			Marshal McKusick and Keith Bostic,
			The Design and Implementation of the 4.4BSD Operating System,
			Addison-Wesley,
			2016
		\bibitem{Linux}
			Robert Love,
			Linux Kernel Development 3/e,
			Addison-Wesley,
			2010,
			ISBN: 978-1-672-32946-3
		\bibitem{Windows}
			Mark Russinovich and Alex Ionescu and David A. Solomon,
			Understanding the Windows I/O System,
			2012
	\end{thebibliography}
\end{document}
