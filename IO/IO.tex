\documentclass[letterpaper,10pt,draftclsnofoot,onecolumn,titlepage]{IEEEtran}

\usepackage{graphicx}
\usepackage{amssymb}
\usepackage{amsmath}
\usepackage{amsthm}
\usepackage{alltt}
\usepackage{float}
\usepackage{color}
\usepackage{url}
\usepackage{enumitem}
\usepackage{pstricks, pst-node}
\usepackage{geometry}
\usepackage{longtable}

\geometry{margin = .75in}

\usepackage{hyperref}

\def\name{Kevin Stine}


\hypersetup{
	colorlinks = true,
	urlcolor = black,
	pdfauthor = {\name},
	pdftitle = {CS444 Project 1},
	pdfsubject = {CS444 Project 1},
	pdfpagemode = UseNone
}

\begin{document}
	\title{\huge Writing Topic 2: I/O and Provided Functionality\\CS 444 Fall 2016}
	\author{\large \name}
	\maketitle
	\newpage
	\section*{I/O}
		Modern operating systems are comprised of many features, joined together by thousands of lines of code, to create the interfaces that we are so used to interacting with.
		Operating systems rely heavily on I/O, also known as Input/Output.
		I/O is fundamental to an operating systems as it allows us to interact with various pieces of hardware such as a keyboard and mouse, and allows us to transfer data across these devices.
		Simple tasks such as typing on a keyboard, playing music over Bluetooth, and getting connected to the WiFi all require some variation of I/O.
		Three of the main modern day operating systems: FreeBSD, Windows and Linux, all utilize I/O in some fashion, however there are various nuances that make them unique to their respective operating system.

		FreeBSD is a Unix-like operating system that descended from Research Unix via the Berkeley Software Distribution.
		Since FreeBSD is an operating system that is Unix-like, it contains a lot of features that are derived from UNIX.
		The basic model of the Unix I/O system is a sequence of bytes that can be accessed either randomly or sequentially, and there are no access methods and no control blocks in a typical Unix user process.
		FreeBSD categorizes hardware devices by being either structured or unstructured.
		Structured devices, also known as block devices, are typically disks, magnetic tapes and include most random access devices.
		On block oriented devices, the FreeBSD kernel supports read-modify-write-type buffering actions to allow reading and writting in a totally random byte-addressed fashion.
		On FreeBSD filesystems are created on block devices.
		Unstructured devices, also known as character devices, are devices which do not support a block structure.
		These can include communication lines, raster plotters and unbuffered magnetic tapes and disks.
		Character devices do typically support large block I/O transfers.
		The BSD kernel introduced an IPC mechanism which is more flexible than pipes and is based on sockets called the Socket IPC.
		A socket is an endpoint of communication referred to by a descriptor.
		Two processes can each create a socket, and then connect those two endpoints to produce a reliable byte stream.
		This new Socket IPC mechanism does essentially what was available previously with the use of pipes without the need for a common parent process to setup the communication channel.
		The transparency of sockets allows the kernel to redirect the output of one process to the input of another regardless if they have a common parent process.
		The socket IPC mechanism can also create a connection between sockets on two unrelated process and it can even connect them on different machines.
		This mechanism widely opens up the functionality of communication between processes however does require extensions to the traditional Unix I/O system calls.
		Previously, the read and write system calls were used for byte-stream type connections, but six new calls were added to allow sending and receiving addressed messages.
		The system calls for writing include send, sendto and sendmsg, and the system calls for reading messages include recv, recvfrom, and recvmesg.
		The FreeBSD kernel used to rely on the traditional read and write system calls, however 4.2BSD introduced the ability to do scatter/gather I/O. \cite{FreeBSD}
		Scatter/gather I/O is where a single procedure call sequentially reads data from multiple buffers and writes it to a single data stream, or reads data from a data stream and writes it to multiple buffers.
		Scatter input uses the readv system call to allow a single read to be placed in several different buffers while the writev system call allows several buffers to be written in a single atomic write.
		Instead of the traditional read and write which passes a single buffer and length parameter, the scatter/gather process passes in a pointer to an array of buffers and lengths, along with a count describing the size of the array.

		The Windows operating system has it's own system for I/O and utilizes it's own I/O manager.
		The I/O manager is the core of the I/O system on Windows because it defines the orderly framework within which I/O requests are delivered to device drivers. \cite{Windows}
		The I/O system on Windows is packet driven and is represented by an I/O request packet called an IRP.
		These IRPs are designed to travel between one I/O system to another, allowing an individual application thread to manage multiple I/O request concurrently.
		An IRP is created in memory by the I/O manager representing an I/O operation.
		A pointer to the IRP is then passed to the correct driver until the completion of the I/O operation.
		Once the I/O operation has completed or needs to be passed on to another driver, the IRP is sent back to the I/O manager to either dispose of, or pass on to another driver.
		In addition to handling IRPs, the I/O manager provides flexible I/O services that allow environment subsystems, such as Windows and Posix to implement their respective I/O functions.
		The I/O manager has no knowledge of anything except for files, and passes the responsibility to translate file-oriented comments such as open, close and read to the driver.
		A typical I/O request starts with an application executing an I/O related function that is processed by the I/O manager, one or more device drivers, and the hardware abstraction layer (HAL).
		On Windows, threads perform I/O on virtual files which refer to any source or destination for I/O that is treated as if it were a file.
		This can be anything from files, directories pipes and mailslots.
		Windows supports two user-mode drivers, the Windows subsystem printer driver which translates device-independent graphic requests to printer-specific commands, and the user-mode driver framework (UMDF) drivers which are hardware device drivers allowed to run in user mode.
		Kernel-mode device drivers can be broken up into file system drivers, plug and play drivers, and non-plug and play drivers.
		File system drivers accept I/O requests to files and satisfy request by issuing their own, more explicit requests to mass storage or network device drivers.
		Plug and play drivers work with hardware and integrate with the windows power manager and PnP (Plug and Play) manager.
		Non-plug and play drivers are drivers or modules which can be loaded that extend the functionality of the system.
		The Windows operating system allows I/O operations to be either synchronous or asynchronous.
		Most I/O operations that applications issue are synchronous, in that the application thread waits while the device performs the data operation and returns a status code when the I/O is complete.
		This allows for the program to continue and access the data immediately after the I/O is complete.
		Asynchronous I/O allows an application to issue multiple I/O requests and continue executing them while the device performs it's I/O operations.
		This can dramatically help improve an application's throughput since it allows the application thread to continue during the I/O operation.
		Similarly to FreeBSD, Windows also allows for the use of scatter/gather I/O through the ReadFileScatter and WriteFileGatther functions.
		Windows uniquely has Fast I/O which is a special mechanism that allows the I/O system to bypass generating an IRP and instead go directly to the device driver stack to the complete the I/O request.

		Similarly to Windows and FreeBSD, Linux provides the abstract machine interface to the user through device drivers.
		The device drivers take a hardware specific interface and map it onto a standard interface within the kernel.
		As with Windows and FreeBSD, Linux has both block devices and character devices.
		Character devices essentially behave as a stream of bytes with no random access provided and once a byte is consumed it cannot be replayed. \cite{Linux}
		These character devices are very straight forward with minimal driver code and straightforward hardware interfaces.
		As with the other operating systems, block devices on Linux are random access, block addressable disk like devices.
		On Linux, the Block I/O subsystem (BIO) works on buffers or memory blocks.
		Each buffer maps to one actual block with a descriptor associated with it known as a buffer head.
		FreeBSD and Linux are very similar in the way that they implement I/O since FreeBSD also utilizes descriptors to reference I/O streams.
		As both FreeBSD and Linux are Unix-like, it makes sense that they would have some similarities in how they implement I/O.
		With FreeBSD and Linux being the most similar in terms of how they implement I/O, each operating system uses functions like read and write to enable the communication between devices.
		Despite their differences, the implementations for I/O are generally pretty even across the board, with some special functions thrown in like the Socket IPC and Fast I/O.
		Even though there may be nuances between special functions, each operating system integrates I/O in some major fashion as it is one of the most fundamental needs of an operating system


	\clearpage

	\begin{thebibliography}{3}
		\bibitem{FreeBSD}
			Marshal McKusick and Keith Bostic,
			The Design and Implementation of the 4.4BSD Operating System,
			Addison-Wesley,
			2016
		\bibitem{Linux}
			Robert Love,
			Linux Kernel Development 3/e,
			Addison-Wesley,
			2010,
			ISBN: 978-1-672-32946-3
		\bibitem{Windows}
			Mark Russinovich and Alex Ionescu and David A. Solomon,
			Understanding the Windows I/O System,
			2012
	\end{thebibliography}
\end{document}
