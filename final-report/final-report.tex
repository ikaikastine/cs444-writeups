\documentclass[letterpaper,10pt,draftclsnofoot,onecolumn,titlepage]{IEEEtran}

\usepackage{graphicx}
\usepackage{amssymb}
\usepackage{amsmath}
\usepackage{amsthm}
\usepackage{alltt}
\usepackage{float}
\usepackage{color}
\usepackage{url}
\usepackage[TABBOTCAP, tight]{subfigure}
\usepackage{enumitem}
\usepackage{pstricks, pst-node}
\usepackage{geometry}
\usepackage{longtable}

\geometry{margin = .75in}

\usepackage{hyperref}

\def\name{Kevin Stine}

\hypersetup{
	colorlinks = true,
	urlcolor = black,
	pdfauthor = {\name},
	pdftitle = {CS444 Final Project},
	pdfsubject = {CS444 Final Project},
	pdfpagemode = UseNone
}

\begin{document}
	\title{\huge Final Paper: Operating Systems Feature Comparison\\CS 444 Fall 2016}
	\author{\large \name}
	\maketitle
	\newpage
	\section*{Operating Systems}
	% Begin Introductions
	Modern day operating systems are complex behemoths built on top of millions of lines of code.
	These operating systems are incredibly complicated, allowing users to do anything from interacting with a mouse and keyboard to using applications all while hiding the internals from the user.
	While most everyday computer users know nothing of the inter-workings of the operating system itself, they expect to be able to interact with their computer in a certain manner and have it function properly.
	From making sure the OS responds to user input, to allowing many processes to run concurrently, the operating system is expected to handle it all.
	Windows, FreeBSD and Linux are three major operating systems that are utilized by consumers, servers, and programs on a daily basis.
	While each operating system has it's own look and feel, they all generally implement similar functionality such as managing input \& output, dealing with processes \& threads, and handling interrupts and synchronization.
	These three topics are imperative for allowing these operating systems to function, and each operating system implements them in their own special way.
	% End Introduction

	% Begin First Topic: I/O and Provided Functionality
	Input \& Output, also known as I/O, is a functionality that is heavily relied on by all modern operating systems.
	I/O is fundamental to an operating systems as it allows us to interact with various pieces of hardware such as a keyboard and mouse, and allows for the transfer of data across these devices.
	Simple tasks such as typing on a keyboard, playing music over Bluetooth, and getting connected to the WiFi all require some variation of I/O.

	FreeBSD is a Unix-like operating system that descended from Research Unix via the Berkeley Software Distribution.
	Since FreeBSD is an operating system that is Unix-like, it contains a lot of features that are derived from UNIX.
	The basic model of the Unix I/O system is a sequence of bytes that can be accessed either randomly or sequentially, and there are no access methods and no control blocks in a typical Unix user process.
	FreeBSD categorizes hardware devices by being either structured or unstructured.
	Structured devices, also known as block devices, are typically disks, magnetic tapes and include most random access devices.
	On block oriented devices, the FreeBSD kernel supports read-modify-write-type buffering actions to allow reading and writing in a totally random byte-addressed fashion.
	On FreeBSD filesystems are created on block devices.
	Unstructured devices, also known as character devices, are devices which do not support a block structure.
	These can include communication lines, raster plotters and unbuffered magnetic tapes and disks.
	Character devices do typically support large block I/O transfers.
	The BSD kernel introduced an IPC mechanism which is more flexible than pipes and is based on sockets called the Socket IPC.
	A socket is an endpoint of communication referred to by a descriptor.
	Two processes can each create a socket, and then connect those two endpoints to produce a reliable byte stream.
	This new Socket IPC mechanism does essentially what was available previously with the use of pipes without the need for a common parent process to setup the communication channel.
	The transparency of sockets allows the kernel to redirect the output of one process to the input of another regardless if they have a common parent process.
	The socket IPC mechanism can also create a connection between sockets on two unrelated process and it can even connect them on different machines.
	This mechanism widely opens up the functionality of communication between processes however does require extensions to the traditional Unix I/O system calls.
	Previously, the read and write system calls were used for byte-stream type connections, but six new calls were added to allow sending and receiving addressed messages.
	The system calls for writing include send, sendto and sendmsg, and the system calls for reading messages include recv, recvfrom, and recvmesg.
	The FreeBSD kernel used to rely on the traditional read and write system calls, however 4.2BSD introduced the ability to do scatter/gather I/O. \cite{FreeBSD}
	Scatter/gather I/O is where a single procedure call sequentially reads data from multiple buffers and writes it to a single data stream, or reads data from a data stream and writes it to multiple buffers.
	Scatter input uses the readv system call to allow a single read to be placed in several different buffers while the writev system call allows several buffers to be written in a single atomic write.
	Instead of the traditional read and write which passes a single buffer and length parameter, the scatter/gather process passes in a pointer to an array of buffers and lengths, along with a count describing the size of the array.

	The Windows operating system has it's own system for I/O and utilizes it's own I/O manager.
	The I/O manager is the core of the I/O system on Windows because it defines the orderly framework within which I/O requests are delivered to device drivers. \cite{Windows}
	The I/O system on Windows is packet driven and is represented by an I/O request packet called an IRP.
	These IRPs are designed to travel between one I/O system to another, allowing an individual application thread to manage multiple I/O request concurrently.
	An IRP is created in memory by the I/O manager representing an I/O operation.
	A pointer to the IRP is then passed to the correct driver until the completion of the I/O operation.
	Once the I/O operation has completed or needs to be passed on to another driver, the IRP is sent back to the I/O manager to either dispose of, or pass on to another driver.
	In addition to handling IRPs, the I/O manager provides flexible I/O services that allow environment subsystems, such as Windows and Posix to implement their respective I/O functions.
	The I/O manager has no knowledge of anything except for files, and passes the responsibility to translate file-oriented comments such as open, close and read to the driver.
	A typical I/O request starts with an application executing an I/O related function that is processed by the I/O manager, one or more device drivers, and the hardware abstraction layer (HAL).
	On Windows, threads perform I/O on virtual files which refer to any source or destination for I/O that is treated as if it were a file.
	This can be anything from files, directories pipes and mailslots.
	Windows supports two user-mode drivers, the Windows subsystem printer driver which translates device-independent graphic requests to printer-specific commands, and the user-mode driver framework (UMDF) drivers which are hardware device drivers allowed to run in user mode.
	Kernel-mode device drivers can be broken up into file system drivers, plug and play drivers, and non-plug and play drivers.
	File system drivers accept I/O requests to files and satisfy request by issuing their own, more explicit requests to mass storage or network device drivers.
	Plug and play drivers work with hardware and integrate with the windows power manager and PnP (Plug and Play) manager.
	Non-plug and play drivers are drivers or modules which can be loaded that extend the functionality of the system.
	The Windows operating system allows I/O operations to be either synchronous or asynchronous.
	Most I/O operations that applications issue are synchronous, in that the application thread waits while the device performs the data operation and returns a status code when the I/O is complete.
	This allows for the program to continue and access the data immediately after the I/O is complete.
	Asynchronous I/O allows an application to issue multiple I/O requests and continue executing them while the device performs it's I/O operations.
	This can dramatically help improve an application's throughput since it allows the application thread to continue during the I/O operation.
	Similarly to FreeBSD, Windows also allows for the use of scatter/gather I/O through the ReadFileScatter and WriteFileGatther functions.
	Windows uniquely has Fast I/O which is a special mechanism that allows the I/O system to bypass generating an IRP and instead go directly to the device driver stack to the complete the I/O request.

	Similarly to Windows and FreeBSD, Linux provides the abstract machine interface to the user through device drivers.
	The device drivers take a hardware specific interface and map it onto a standard interface within the kernel.
	As with Windows and FreeBSD, Linux has both block devices and character devices.
	Character devices essentially behave as a stream of bytes with no random access provided and once a byte is consumed it cannot be replayed. \cite{Linux}
	These character devices are very straight forward with minimal driver code and straightforward hardware interfaces.
	As with the other operating systems, block devices on Linux are random access, block addressable disk like devices.
	On Linux, the Block I/O subsystem (BIO) works on buffers or memory blocks.
	Each buffer maps to one actual block with a descriptor associated with it known as a buffer head.
	FreeBSD and Linux are very similar in the way that they implement I/O since FreeBSD also utilizes descriptors to reference I/O streams.
	As both FreeBSD and Linux are Unix-like, it makes sense that they would have some similarities in how they implement I/O.
	With FreeBSD and Linux being the most similar in terms of how they implement I/O, each operating system uses functions like read and write to enable the communication between devices.
	Despite their differences, the implementations for I/O are generally pretty even across the board, with some special functions thrown in like the Socket IPC and Fast I/O.
	Even though there may be nuances between special functions, each operating system integrates I/O in some major fashion as it is one of the most fundamental needs of an operating system
	% End First Topic: I/O and Provided Functionality

	% Begin Second Topic: Interrupts and Synchronization
	In addition to I/O, modern operating systems utilize interrupts in order to actually process the input \& output that is sent to the computer from external hardware devices.
	Most computers have many devices such as a keyboard, mouse, hard drive, printer and scanners plugged into them.
	These various devices all need to have access to the CPU, however we are unable to predict exactly when it will need that access.
	Because of this, modern operating systems utilize interrupts, a way in which the CPU can find out which devices need attention and when.
	An interrupt is a signal to the processor that is emitted by hardware or software indicating an event that needs immediate attention.
	Interrupts are synchronous with respect to the current process meaning that interrupts can happen at any time.
	They work by giving each device an interrupt line that it can use to signal the processor and when the interrupt is signaled, the processor executes a routine called an interrupt handler to deal with the interrupt.
	The benefit of interrupts is there is no overhead when there are no requests pending.

	FreeBSD follows the pattern of other multi-threaded Unix kernels and deals with interrupt handles by giving them their own thread context.
	Giving an interrupt handler it's own thread context allows for the interrupt handler to block on locks and are run at real-time kernel priority.
	In FreeBSD, the interrupt threads are referred to as heavyweight interrupt threads because switching to an interrupt thread involves a full context switch.
	Currently, the kernel in FreeBSD is preemptive, however they only preempt a kernel thread when they release a sleep mutex or when an interrupt comes in.
	Having a fully preemptive kernel is in the pipeline for the future FreeBSD kernel.
	FreeBSD also supports Message Signaled Interrupts (MSIs) starting with PCI 2.2. This allows each PCI function to have one or more interrupt messages rather than each PCI function being restricted to one.
	FreeBSD implements MSI messages as sys\_res\_irq interrupts. The PCI bus driver is responsible for programming the various MSI registers.
	The format of the message address and data fields is platform specific.
	For x86 platforms, the x86 MSI code uses interrupt source objects to manage IRQs for MSI messages and provides a single struct pic shared by all MSI interrupt sources.
	Once an MSI interrupt source is created in FreeBSD, it is never destroyed, but may be reused by a different device if it is released by a driver and another driver makes an allocation request.

	Windows utilizes the Windows Driver Framework (WDF) which allows for drivers to handle a devices hardware interrupts. \cite{Windows}
	In order for a WDF driver to handle a device's hardware interrupts, it must first create a framework object for each interrupt that each device can support.
	In the most recent versions of Windows (8 and newer), Kernel-Mode Driver Frameworks (KMDF) and User-Mode Driver Frameworks (UMDF) drivers can create interrupt objects requiring passive-level handling.
	Passive-level handling is used in specific cases such as System on a Chip (SoC) platforms, but most drivers should utilize DIRQL interrupt objects.
	Windows also supports Message-signaled interrupts (MSIs) which is an alternative in-band method of signaling an interrupt.
	For MSIs, the driver must create a framework interrupt object for each interrupt vector or MSI message that the device can support.
	Servicing an interrupt in Windows requires you to save volatile information such as register contents quickly and process the saved volatile information in a deferred procedure call (DPC).
	Interrupts in Windows are centered around the Interrupt Service Routing (ISR). These are what is used to service the interrupts and the system calls the ISR each time it receives that interrupt.
	Windows supports two different types of ISRs.
	The driver can register an InterruptService routine to handle line-based or message-signaled interrupts where the system passes a driver-supplied context value.
	The driver can also register an InterruptMessageService routine to handle message-signaled interrupts where the system passes both a driver-supplied context value and the message ID of the interrupt message.
	DIRQL interrupts require the ISR to run while holding a driver-supplied spin lock so that the ISR prevents additional interrupts while it saves volatile information.
	This helps to ensure that the volatile information is not lost.

	Similarly to Windows and FreeBSD, Linux also uses interrupts to asynchronously give CPU time to devices. \cite{Linux}
	In Linux, the interrupt mask registers are write only meaning that you must keep a local copy of what the mask registers are set to.
	Linux modifies these saved masks in the interrupt enable and disable routine and writes the full masks to the registers every time.
	When an interrupt is signaled, the interrupt handling code reads the two interrupt status registers (ISRs), not to be confused with the Interrupt Service Routing in Windows.
	In Linux, the kernel's interrupt handling data structures are setup by the device drivers as they request control of the system's interrupts.
	Individual device driers call these routines to register their interrupt handling routine addresses.
	Linux also allows for device drivers to probe for their interrupts. Once a device driver causes a device to interrupt, all of the unassigned interrupts in the system are enabled.
	To handle interrupts, Linux uses a set of pointers to data structures that contain the address of the routines that handle the system's interrupts.
	This is crucial as the Linux interrupt handling subsystem must route the interrupts to the right pieces of handling code.
	Each device driver is responsible for requesting the interrupt that is wants when the driver is initialized.
	Similarly to FreeBSD, the Linux interrupt handling code is architecture specific.
	When an interrupt occurs, Linux reads the interrupt status register to determine the source. It then translates the source into the irq\_action vector.
	Linux will check to see if there is an interrupt handler for that interrupt, and if there is, it will call into the interrupt handling routine for all of the irqaction data structures.
	As with Windows and FreeBSD, Linux also has support for message signaled interrupts.

	Rather than using the polling method, Windows, FreeBSD and Linux all utilize interrupts to handle devices that need CPU time.
	While the different operating systems may have different meanings for certain acronyms, (ISRs in Linux and Windows), they generally all implement interrupts in a similar fashion.
	Each operating system supports Message Signaled Interrupts and has it's own version for handling interrupts.
	FreeBSD and Linux differ from Windows because interrupt handling for FreeBSD and Linux is architecture specific.
	Windows allows for two different types of interrupt service routing, while Linux just uses a set of pointers to data structures that contain the address of the routines.
	% End Second Topic: Interrupts and Synchronization

	% Begin Third Topic: Processes and Threads
	Most modern operating systems are built utilizing processes and threads.
	In order to maximize the efficiency of handling these processes and threads, operating systems rely on CPU scheduling to monitor their behavior.
	Three of the major modern operating systems: Windows, FreeBSD, and Linux all utilize processes, threads and scheduling.
	A process provides the resources which are needed to execute a program.
	This would include items such as the virtual address space, executable code, unique processes identifiers, environment variables, and at least one thread of execution.
	A thread is an entity within a process that can be scheduled for execution. These threads maintain exception handlers, scheduling priorities and a unique thread identifier. Multiple threads are able to exist within one process, each executing concurrently and sharing resources.
	These threads are managed independently by the scheduler of the operating system.
	CPU scheduling is utilized to keep all the computer resources busy or to allow multiple users to share system resources effectively.
	Each of the three respective operating systems implement some way of managing processes, threads and scheduling.

	Microsoft Windows supports preemptive multitasking, a concept which creates the effect of simultaneous execution of multiple threads from multiple processes. While multiple threads from multiple processes are not actually capable of running simultaneously, the OS does a great job at hiding that fact.
	To the user, they see multiple windows open, multiple programs running and can assume that everything is just working simultaneously when in fact, the OS is working hard to provide each processes with a fair amount of time on the CPU.
	Within Windows, threads are scheduled to run based on their scheduling priority, which ranges from zero (lowest priority) to 31 (highest priority).
	Despite having a zero priority, only the zero-page thread is allowed to have a priority of zero. Windows also utilizes the round-robin method for process scheduling.
	The round-robin scheduler utilizes time-sharing, giving each job a time slice which is the amount of time it is allowed on the CPU.
	If the job is not completed when the time slice is up, the job is interrupted and resumed once another time slice is assigned to that process.
	Windows treats all threads with the same priority as equal, and begins by assigning time slices in a round-robin fashion to all threads with the highest priority.
	If those processes are not ready to run, the system moves to the next highest priority processes and assigns them a time slice.
	Windows creates a base priority, which is a combination of the process priority class and the thread priority level.
	The process priority classes can be idle, below normal, normal, above normal, high priority, and real time.
	The thread priorities can be idle, lowest, below normal, normal, above normal, highest and time critical.

	On the other hand, FreeBSd implements a few things in a different fashion. In FreeBSD, all threads that are runnable are assigned a scheduling priority which determines the run queue they are placed in.
	The system scans the run queues from highest to lowest priority and chooses the first thread on the first non empty queue.
	If there are multiple threads on the queue, the system will run the threads in a round robin fashion.
	If a thread blocks, it is not put back onto any of the run queues and if a thread uses up the time quantum, it is placed as the end of the queue.
	FreeBSD has found that the longest quantum that could be used without loss is 100ms.
	It also uses a time-share thread scheduling algorithm which is based on multilevel feedback queues.
	The system will adjust the priority of a thread dynamically to reflect resource requirements and the amount of resources consumed by the thread.
	If a thread that is not currently running is assigned a higher priority than the currently running thread, the system switches to that thread immediately if in user mode, otherwise it switches to the higher-priority thread once the current thread exits the kernel.
	This system favors interactive jobs by increasing the scheduling priority of threads that are blocked waiting for I/O for 1 or more seconds, and lowers the priority of threads that utilize a lot of CPU time.
	Threads are given a certain level of priority, between the range of -20 and 20 with the normal value being 0.
	Negative values increase the thread’s priority while positive values decrease them.

	Linux also uses preemptive multitasking where the scheduler decides when a process is to cease running and a new process is to resume running.
	The scheduler used by Linux is the O(1) scheduler since the scheduler can run in constant time. Similarly to FreeBSD, Linux sets the priority on a scale from -20 to 19 with the default priority being zero.
	The more negative the values are, the less nice they are, which makes them the highest priority. The Linux scheduler dynamically determines the timeslice of a process based solely on its priority.
	This allows for higher priority processes to run longer and more often than lower priority processes.
	A process with a nice value of -20 receives the maximum timeslice and a process with a nice value of 19 receives the minimum timeslice.
	In Linux, the minimum timeslice is 10ms, the default time slice is 150ms, and the maximum timeslice is 300ms.

	Windows, FreeBSD, and Linux all have their own implementations for handling processes, threads and scheduling.
	Both Windows and Linux utilize a type of preemptive multitasking and while FreeBSD can preempt threads executing in kernel mode, it is generally not done.
	Each of the three operating systems use a round-robin algorithm for handling all processes without a priority.
	Round robin seems to be a relatively simple and easy to implement algorithm which prevents the issue of starvation.
	Since all three operating systems use time slices, it makes sense that they would all use round robin.
	Setting the priority of threads and processes differs between Windows and Linux.
	On Windows, the priority is determined by a combination priority class of the process and the priority level of the thread, giving a unique number on the scale from zero to thirty-one.
	On FreeBSD and Linux, the priority is measured on the nice scale, from -20 to 19 (Linux) or 20 (FreeBSD).
	The amount of time used for the timeslice is very different between FreeBSD and Linux. FreeBSD uses a constant 100ms for their time slice, while Linux determines the time slice dynamically, based on the process priority and ranges from 10ms to 300ms.
	It seems like across the board, round-robin is the best choice for scheduling threads without a priority.
	From there however, the three operating systems take different routes on how to determine priority, and how to determine the time slice that each thread will be given.
	% End Third Topic: Processes and Threads

	% Begin Conclusion
	With each operating system comes it's own flair on how their internal pieces are designed.
	Whether it be setting the timeslice to a specific time or implementing I/O a particular way, each operating system is uniquely designed that makes it special in comparison to other operating systems.
	Getting a chance to research Windows, FreeBSD and Linux has been really interesting and opened some doors into areas that I previously had not explored.
	It's easy for most people to just use a computer, such as knowing that when they type on their keyboard, characters will appear on the screen.
	Getting to take a peek behind the scenes at how the operating system manages the keyboard by using interrupts has really opened my eyes to the depths of the Kernel and operating systems in general.
	Learning how drivers make use of interrupts so that they can interact with the CPU has given me more perspective on how applications interact with the rest of the OS.
	All of this research has also made me realize just how complex computers are, and everytime I open up my laptop I begin thinking about all the processes that are being run, how they've utilized their timeslice, or somethings beeg interrupted because I began typing.
	For a long time the operating system has just been a "black box" to me where I knew how to interact with it, but didn't really know what was going on behind the scenes.
	While there is plenty that I'm still unaware of, I've started to get a glimpse into the magic behind what makes an operating system work.


	% End Conclusion


	\clearpage

	\begin{thebibliography}{3}
		\bibitem{FreeBSD}
			Marshall Kirk McCusick and George V. Neville-Neil,
			Design and Implementation of the FreeBSD Operating System 2/e,
			Addison-Wesley,
			2015,
			ISBN: 978-0-32196897-5
		\bibitem{Linux}
			Robert Love,
			Linux Kernel Development 3/e,
			Addison-Wesley,
			2010,
			ISBN: 978-1-672-32946-3
		\bibitem{Windows}
			Windows Dev Center,
			About Processes and Threads,
			Microsoft Corporation
	\end{thebibliography}
\end{document}
