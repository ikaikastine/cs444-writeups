\documentclass[letterpaper,10pt,draftclsnofoot,onecolumn,titlepage]{IEEEtran}

\usepackage{graphicx}
\usepackage{amssymb}
\usepackage{amsmath}
\usepackage{amsthm}
\usepackage{alltt}
\usepackage{float}
\usepackage{color}
\usepackage{url}
\usepackage{balance}
\usepackage[TABBOTCAP, tight]{subfigure}
\usepackage{enumitem}
\usepackage{pstricks, pst-node}
\usepackage{geometry}
\usepackage{longtable}

\geometry{margin = .75in}

\usepackage{hyperref}

\def\name{Kevin Stine}

\hypersetup{
	colorlinks = true,
	urlcolor = black,
	pdfauthor = {\name},
	pdftitle = {CS444 Final Project},
	pdfsubject = {CS444 Final Project},
	pdfpagemode = UseNone
}

\begin{document}
	\title{\huge Final Paper: Operating Systems Feature Comparison\\CS 444 Fall 2016}
	\author{\large \name}
	\maketitle
	\newpage
	\section*{Operating Systems}
	% Begin Introductions
	Modern day operating systems are complex behemoths built on top of millions of lines of code.
	These operating systems are incredibly complicated, allowing users to do anything from interacting with a mouse and keyboard to using applications all while hiding the internals from the user.
	While most everyday computer users know nothing of the inter-workings of the operating system itself, they expect to be able to interact with their computer in a certain manner and have it function properly.
	From making sure the OS responds to user input, to allowing many processes to run concurrently, the operating system is expected to handle it all.
	Windows, FreeBSD and Linux are three major operating systems that are utilized by consumers, servers, and programs on a daily basis.
	While each operating system has it's own look and feel, they all generally implement similar functionality such as managing input \& output, dealing with processes \& threads, and handling interrupts and synchronization.
	These three topics are imperative for allowing these operating systems to function, and each operating system implements them in their own special way.
	% End Introduction

	% Begin First Topic: I/O and Provided Functionality


	% End First Topic: I/O and Provided Functionality

	% Begin Second Topic: Interrupts and Synchronization


	% End Second Topic: Interrupts and Synchronization

	% Begin Third Topic: Processes and Threads


	% End Third Topic: Processes and Threads

	% Begin Conclusion


	% End Conclusion


	\clearpage

	\begin{thebibliography}{3}
		\bibitem{FreeBSD}
			Marshall Kirk McCusick and George V. Neville-Neil,
			Design and Implementation of the FreeBSD Operating System 2/e,
			Addison-Wesley,
			2015,
			ISBN: 978-0-32196897-5
		\bibitem{LinuxKernelDev}
			Robert Love,
			Linux Kernel Development 3/e,
			Addison-Wesley,
			2010,
			ISBN: 978-1-672-32946-3
		\bibitem{Windows}
			Windows Dev Center,
			About Processes and Threads,
			Microsoft Corporation
	\end{thebibliography}
\end{document}
